\chapter{Discussion}\label{chap:discussion}
In this work the three specific goals stated in the introduction are recalled as:
\begin{itemize}
	\item[1.] Generation of a robust and reliable energy consumption data set.
	\item[2.] Estimating the direct impact of consumed electricity on CO$_2$ footprint.
	\item[3.] Use well defined \acrfullpl{kpi} to identify optimization potential in energy consumption.
\end{itemize}
To accomplish the first goal, in this word a methodology to create a realistic live data set of an actual office environment including hardware, software and performance indicators was defined. The goal was achieved by uniformly distribute sensing hardware into a live office environment. By an automated request and pooling program, requested data was accumulated with sufficiently high density in order to increase signal to noise ratio by averaging over fixed intervals. While basic consumption data was collected with faults, see section \ref{sec:datadisscussion}, the order of magnitude can be very accurately determined, and broadly match expected values from the literature.\\
By evaluating modern CO$_2$ estimation protocols, the risk of over fitting CO$_2$ figures through recent, but misleading values was identified, and a conservative estimation was given. By finding a better data source, like increased mandatory reporting of CO$_2$ contributions of all suppliers to the experiments devices, the average estimation should be supplanted by more recent and dynamic methods.\\
The indications in the data using specified \acrshortpl{kpi} was defined in section \ref{sec:metrics}. The evaluation of the idle time metric has shown valuable specific insights in PC and Monitor consumption, while showing only mixed applicability for device agnostic measurements. The modification of parameters according to device specifications is needed to increase reliable indication on energy usage by all kinds of devices. Meanwhile the introduction of \acrfull{wwcr} has shown great promise in differentiating consumers of similar total energy consumption into a ranking of little to great possible potential in energy conservation. While producing multiple positive results, different approaches can be considered to improve the quality of data generation and validation. The individual limitations of the experiment that became obvious during the hardware setup process were:

\paragraph{Controlled environment}is, while working against the idea of creating realistic values, a well defined environment where parameters can be changed if necessary have many benefits. From missing data due to lack of adherence to the usage of sensors, to the fluctuating quality in network capacity, up to the unavailability of certain possibilities, like access to the service behind a firewall. Components that are permanently out of reach for any type of experiment greatly hinder possibilities and have to be accounted for in advance.

\paragraph{Blinding experiments}of this kind will likely increase the veracity of the data. During experimental process and measurement, multiple office workers showed aversion to the prospect of being monitored. While missuses and repercussions can be avoided using privatization and pseudonymization techniques, the knowledge and physical reminder in form of a visible sensor is most likely influencing staff behavior. While this fact could be useful during interventions, a neutral data gathering will be influenced by this.

\paragraph{Counter measurements}using laboratory equipment might be a useful tool to recognize and minimize faulty live data. As the experimenter has no (permanent) immediate connection or access to the devices, it is quite difficult to recognize unrealistic values without further information on the device. For example in the case of this measurement the ratios of consumption due to computation and monitoring match with the estimated ratios found in the related works in section \ref{chap:relatedworks}, however due to the advancements in computing power and flat screen monitoring, it is unclear if this ratio should still hold.

\paragraph{Improved real time CO2 footprint estimation}would be a great asset when the goal is to more accurately react to specific times where energy conservation should be prioritized, as when renewable energies are more readily available during seasons. While there are different API provider giving accurate daily information to renewable energy generation \cite{co2_footprint_api} in a respective region, due to the interconnected grid system it is fundamentally impossible to determine which energy you are using currently. This has to be determined by retroactively reevaluating energy balance and trade between countries(refer to section \ref{sec:co2works} for specifics) and can only be estimated in advance. Another possibility would be a bypass of the grid by using locally generated electricity. As this kind of power would most likely be renewable, the value of this feature in power conservation would still need to be determined.

\pagebreak
\paragraph{Incorporating independent measurements}ranging from accurate count of human traffic, self reliant ways to determine efficiency and value of work per energy consumed, or complementing electricity data, like electric bills could improve overall veracity of the key indicators that are at the core of this experiment.

\section{Data Discussion}\label{sec:datadisscussion}
During data collection separate evaluations seemed to result in contradicting data. As seen in the visualization of user 7 in figure \ref{fig:user_comparison} and his respective result table, an obvious sensor bias compiled to a significant error. Mixed with the expected measurement noise mainly due to voltage fluctuations, refer to section \ref{sec:vflux}, and a completely unused workplace, the negative consumption summed to a significant reduction in power consumption overall. Utilizing more controlled environments, it needs to be evaluated if simply ignoring a biased sensor, by cutting negative values, if a production of power by the device is categorically impossible, or if adding similar noise to the signal reaches a more accurate result to actual consumption values. Overall consumption and ratios in processing vs. visualizations stayed within the expected range found in the literature. The evaluation using the proposed \gls{kpi} \acrfull{wwcr} gave some insight to differentiate devices and users with possible bigger efficiency gains by experiencing higher off work consumption. To find applicable intervention methods the metric should be evaluated further in future experiments. Using a simple definition of device Idle Time across device types did only prove successful, if the consumption profile of the device matched with the definitions. Especially in the case of multiple devices per sensor, and high energy intensive printer the idle time did not show useful information, as the threshold was passed seldom. The application of CO$_2$ footprint estimations gives a rough estimate on the impact on greenhouse gas emissions, but the efficacy has to be determined using annual report data by the power provider and distributors.

\section{Future Work}
While providing a solid data basis for a real office environment. The data collection is generally only the first step in making an impact in the environment. There are different possibilities to enlarge the scope of this work to improve its usefulness.
\paragraph{Change of context}while office spaces contribute a proper partition of total energy consumption. The service sector only contributes 25\% of overall electricity demand. This evaluates to the third most sector behind industrial 43\% and residential 26\% \cite{enerdata}. To get a chance of meeting environmental sustainability goals as described in section \ref{sec:sustainability}, all other sectors above need at least as much research to tackle the energy conservation goals.
\paragraph{Widen the scope}of office context measurements, and develop concrete aids and methodologies to influence consumption behavior wherever possible. This includes on the one hand a diversification of different kinds of office spaces in different cities and environments. On the other hand, other aspects of IT, service and communications like data centers, AI research and communication infrastructure need to be included to complete the picture of electricity consumption in the office world.
\paragraph{Adapt the current measurement model} to facilitate other kinds of research. A valuable contribution could be a thorough labeling of the data set to enable the use of AI models to be trained on real consumption data, with the aim to facilitate for example narrow artificial intelligence to find solutions to energy optimization problems.
\paragraph{Find a way to privatize}energy consumption measurements. Modern tech companies frequently find ways to encourage individuals to gather data for a specific tech service to use. The examples range from traffic data using phone location to health data in order to receive an insurance bonus. By providing a service to the public, like a tool to enable smart grid exchange between citizens using localized renewable energy generators, when consumption data is freely given, the gathering of data could become a secondary task if a valuable incentive can be given to prospective users. 