% !TeX spellcheck = en_US
% !TeX encoding = UTF-8
\chapter*{Abstract}
Due to progressively tightening climate conservation legislation, energy efficiency becomes an increasingly important issue in the context of office work. By the year 2050 the Federal Climate Change Law mandates negative greenhouse emissions, which is most likely only achievable by a massive reduction in emissions due to energy generation. By creating an experimental measurement laboratory in a real office environment, this paper describes the creation of a reliable data source and evaluation methodology. The data set contains a time series of power consumption of different device types. The gathered data differentiates various device types and workplaces. Three discrete key performance indicators (CO$_2$ Footprint, Workday to Weekday Consumption Ratio, Idle Time) are used to evaluate the established data basis and results are presented in various visualizations. Additionally an interactive dashboard tool to further explore the data set is introduced.